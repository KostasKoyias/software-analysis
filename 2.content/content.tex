\documentclass{article}
\usepackage[utf8]{inputenc}
\usepackage[margin=80pt]{geometry}
\usepackage{amssymb}
\usepackage{amsmath}
\usepackage{enumitem}
\usepackage{graphicx}
\usepackage{algorithmic}
\usepackage[table,xcdraw]{xcolor}
\usepackage[hidelinks]{hyperref}
\usepackage{xcolor}
\newcommand\LocalVar{{\color{blue} Local Var }}
\newcommand\Initialize{{\color{blue} Initialize }}
\newcommand\Call{{\color{blue} Call }}
\hypersetup{
    colorlinks=true,
    linkcolor=black,
    filecolor=magenta,      
    urlcolor=blue,
}
\newlength\myindent
\setlength\myindent{2em}
\newcommand\bindent{%
  \begingroup
  \setlength{\itemindent}{\myindent}
  \addtolength{\algorithmicindent}{\myindent}
}
\newcommand\eindent{\endgroup}

\begin{document}
\tableofcontents
\newpage
\section{Introduction}
This section aims to clarify the goal of the project, as well as describing the philosophy and the tools used by this team to efficiently collaborate and achieve better results. In addition, any assumptions made about this project are reported and a short description of all content is given.
\subsection{Goal of the project}
Given the description of a software application, an e-broker\footnote{An e-broker is a brokerage house that allows you to buy and sell stocks and obtain investment information from its Web site} system, use a variety of diagrams(UML, Dataflow etc.) in order to better analyze the requirements, functional or not, of the application. 
\subsection{How we worked}

\subsubsection{Trello}
To better collaborate and organize our work, we used \href{https://trello.com}{\underline{\textit{Trello}}}, a flexible project management tool, by assigning weekly tasks to each member of the team.

\subsubsection{Git}
In order to exploit work parallelism, while keeping track of all progress the rest of the team made in the meantime, we used the famous VCS \href{https://github.com/}{\underline{\emph{git}}}\footnote{One can take a look at the source code of this project or even contribute, \href{https://github.com/KostasKoyias/Software_Analysis}{\underline{\textit{here}}}} using \href{https://nvie.com/files/Git-branching-model.pdf}{\underline{\emph{this}}} branching model. For those unfamiliar with git, it is a distributed version-control system, tracking changes in source code during software development to better coordinate work among programmers. Whenever a task was completed, the corresponding member requested a review. Then, the rest of the team was able to review the set of changes, discuss potential modifications, or even enhance the commit. Eventually, in most cases, the request was approved and merged into the main branch.

\subsubsection{Discord}
As in every team project, in order to optimize performance, meetings really were a necessity. We sure had some face-to-face meetings, but that was not always possible. So, another tool we used is \href{https://discordapp.com/}{\underline{\emph{Discord}}}, a platform designed for video gaming communities, that specializes in text, image, video and audio communication between users in a chat channel. Using Discord's video-chatting and screen-sharing features helped us co-ordinate and co-operate easily and fast. As a team, we agreed on having short but frequent meetings(2 or 3, 10 minute meetings per day) in order to take full advantage of time, optimizing performance in general.

\subsection{Submission}
Work load was pretty well distributed amongst all members of this team throughout the whole process of deployment. The one member responsible of reviewing the final version of this pdf and submitting it to e-class is{\hypersetup{
    urlcolor=black,    
}
\href{https://github.com/KostasKoyias}{\emph{Konstantinos Koyias}}}.  

\newpage
\subsection{Assumptions}
Despite the detailed and informative description of the project, there were a few points we had to intervene, make our own assumptions and decide on what is best to follow. The most important of those were:
\begin{itemize}
\item A network partition needs to be included in possible scenarios.\\
Take a look at {\hypersetup{
    linkcolor=blue,    
}\hyperlink{CmdTrns}{\textit{Command Transmission}}} for more. 
\item A client does not have to wait for a command report, it will be delivered at random time. Meanwhile a client can
carry on navigating or issuing another command. The same holds for the e-Broker System forwarding a command to the
Athens Stock Exchange. Both of these requests are asynchronous. Take a look at the UML {\hypersetup{
    linkcolor=blue,    
}\hyperlink{Asynchronous}{\textit{Sequence Diagram}}} for more.
\end{itemize} 

\subsection{Chapters} 

\subsubsection{Structured Analysis}
This chapter contains:
\begin{itemize}
\item The dataflow diagram of the e-Broker system.
\item Dataflow diagrams for levels 1 and 2 of procedure decomposition.
\item The process decomposition tree.
\item Specifications for:
\begin{enumerate}
\item The process of transmitting order in Structured English.
\item The process of supply estimation using a Decision Table.
\item The process of a client's log in the e-Broker system using a Tree.
\item Removing a command from the Data Dictionary.  
\end{enumerate}
\end{itemize} 

\subsubsection{UML}
In this chapter, we used the general-purpose, developmental, modeling language UML in order to visualize the design of the e-Broker system. Specifically, the following were included:

\begin{itemize}[{label=\tiny$\triangleright$}]
\item Use Case diagram of the system.
\item Class diagram of the system.
\item State-machine diagram of the entity ``Command''.
\end{itemize}

To better model the stock selling/buying operations we included:

\begin{itemize}[{label=\tiny$\triangleright$}]
\item A main success case Scenario and some alternative ones.
\item An Activity diagram.
\item A detailed Class diagram.
\item A Sequence Diagram.
\item A Communication Diagram.
\end{itemize}

\subsubsection{Structured Design}
This is the chapter where a systematic methodology using
\begin{itemize}
\item a Program Structure diagram and 
\item a piece of pseudo-code that formulates the control unit of
\begin{itemize}[{label=$\circ$}]
\item the main transformation 
\item the M3 transformation 
\item the M4 transformation
\end{itemize}
and the presentation unit of
\begin{itemize}[{label=$\circ$}]
\item the M2 transformation 
\end{itemize}
for the corresponfing Program Structure diagram.
\end{itemize}
determines the design specifications of the application. 

\newpage
\section{Structured Analysis}
Let us try and understand how the system works in a logical, using a systematic, graphical approach, Structured Analysis. 

\subsection{e-Broker Dataflow diagram}
A simple, overall, graphical representation of e-Broker is the one below\\
\hspace*{5mm}\includegraphics[scale=0.8]{dataflow_0.png}\\
Level 0 is the simpliest of all dataflow diagrams. A client fills the log in form and his access rights are specified
using the database on the right. Then he issues a command(buy/sell)
which is forwarded to the A.S.T and as a result, he gets a report.


\subsection{Procedure decomposition Dataflow diagrams}
Now let us try and decompose the above diagram a bit, to better isolate each unit and process of the system,
taking a more precise look on each of them seperately.
\subsubsection{Single level decomposition}
\includegraphics[scale=0.8]{dataflow_1.png}\\
Notice how the Log in unit passes on the authorization data,
which include all information about that particular client to the Transfer
unit so that the transaction can be completed.

\newpage
\subsubsection{Double level decomposition}
As you might thought by reading the description of this system, the Transfer unit can be broken down to
a \textbf{Form} and a \textbf{Transmission} unit, which include what their names imply. So, the most detailed diagram version
for the e-Broker system looks as follows\\
\hspace*{8mm}\includegraphics[scale=0.8]{dataflow_2.png}\\
Notice how two \textbf{regular files} come into the game in this very last diagram. One for the form
defining the transaction details and the other for the report describing the outcome of the trasnsaction.

\subsection{Process decomposition Tree}
Even though it is implied by the diagrams above, a nice graphical representation in the form of a tree
of how complex processes decompose to smaller, isolated and more simple is certainly useful.
\hspace*{20mm}\includegraphics[scale=0.8]{dataflow_3.png}\\

Notice how the leaves of this tree manage to get a \textbf{single} job done and can not be further decomposed. 

\subsection{Specifications}
In this section, each specification is examined separately using a variety of Structured Analysis tools. 

\subsubsection{Transmitting Order}
Using Structured English, the Transmitting Order process from the e-Broker system point of view looks as follows:\\
\begin{algorithmic}[H]
 \WHILE{Form Is Invalid OR Customer is Re-editing}
  	\STATE Display Form
  	\STATE On Submit do
  	\IF{Form Is Incomplete}
  		\STATE Display ``All fields required" message
   
    \ELSIF{There is Field with Invalid Type}
  		\STATE Display ``Invalid field value" message
  	\ELSE
  		\STATE Display Preview
  		\STATE Display ``Back to Editing?" message
  	\ENDIF
 \ENDWHILE
 \STATE Forward command to A.S.T
 \STATE Get response
 \IF{Response is negative}
  	\STATE Display ``Rejected" message
 \ELSE
  	\STATE Display ``Approved" message
 \ENDIF

\end{algorithmic}

\subsubsection{Supply Estimation}
In this little section, we will use a decision table in order to help us simplify the complex process of calculating the exact charge of a Customer for a transaction.
\begin{table}[htbp]
\begin{tabular}{|l|l|l|l|}
\hline
\rowcolor[HTML]{EFEFEF} 
\textbf{Conditions} & Rule 1               & Rule 2               & Rule 3               \\ \hline
Volume              & $0 - O_1$            & $O_1-O_2$            & $O_2-O_3$            \\ \hline
\rowcolor[HTML]{EFEFEF} 
\multicolumn{4}{|l|}{\cellcolor[HTML]{EFEFEF}\textbf{Actions}}                           \\ \hline
Set charge equal to & $P_1$\% $\times Volume$ & $P_2$\% $\times Volume$ & $P_3$\% $\times Volume$ \\ \hline
\end{tabular}
\end{table}

\newpage
\subsubsection{Log in}
Authentication is not always simple.\\
This is a detailed \textbf{decision tree} that proves just that.\\
\begin{center}
\hspace*{5mm}\includegraphics[scale=0.4]{decisionTree}
\end{center} 

\subsubsection{Data dictionary of Command Report}
For each element included in the Command Report, one can find information about
its format data type and other meta-data in the table below. 
\begin{table}[htbp]
\begin{tabular}{|l|l|l|}
\hline
\multicolumn{3}{|l|}{\Large \textbf{Data Dictionary}}      \\ \hline
\textbf{Name}  & \textbf{Description}           & \textbf{Type}         \\ \hline
SN  & Stock Name            & nvarchar(50) \\ \hline
TT  & Transaction Kind      & boolean      \\ \hline
PR  & Price                 & float        \\ \hline
TM  & Time                  & date         \\ \hline
N   & Stocks Number         & integer      \\ \hline
OPR & Total/Overall Price   & float        \\ \hline
ON  & Overall Stocks Number & integer      \\ \hline
SP  & Supply Estimation     & float        \\ \hline
\end{tabular}
\end{table}\\
Notice that Transaction\_Kind is of type boolean, where true corresponds to a buy and false to a sale. 


\newpage
\section{UML}
In this chapter, all neccessary UML diagrams for the e-Broker system will be created. In order to distinquish relationship types, we used this map: \\\hspace*{16mm}\{always: include, might: extend, generic: generalization\}. \\
This gets a bit more clear in the following subsection. 
\subsection{Use Cases}
After taking a good look at the description of the system, we came up with the following actors:
\begin{itemize}
\item Customer(Primary)
\item Customer Support Department(Secondary)
\item Athens Stock Exchange(Secondary)
\end{itemize}
 and the corresponding use cases. Each customer can:
\begin{itemize}
\item \textbf{log in}\\This action will \textbf{always} trigger a password check, and it 
\textbf{might} lead to an error message. In this case a \textbf{generic} ``forgot password'' request \textbf{might} happen, which can be either a password reminder via an e-mail, or a call to the \textit{C.S.D} for a new password or/and a command to be executed.
\item \textbf{command}\\A \textbf{generic} command, can be either a creation or a cancellation, both of which will \textbf{always} have to be reviewed by the \textit{A.S.E}. This \textbf{generic} review will either lead to an approval or a rejection. An approved command \textbf{might} at some point get cancelled, otherwise the \textit{Customer} will 
\item \textbf{get a report}
\end{itemize}
so we came up with the following use case diagram\\
\begin{center}
	\includegraphics[scale=0.3]{use_cases}
\end{center}

\newpage
\subsection{Classes}
This is the section, where classification took place. Notice that C.S.D and A.S.E are just objects, like instances, not classes, so they have no place in the following diagram.\\
\begin{center}
	\includegraphics[scale=0.5]{classesII}\\
\end{center}
Let us now explain what is happening on the diagram above.\\
\begin{itemize}
\item First off, some data is kept for each \textbf{Customer}, a \textbf{Cash} and a \textbf{Title}. 
These, can not really exist without a Customer. 
This is called composition and we used a black diamond to denote it.
\item Each Customer is able to request for as many \textbf{Command}s as he wishes, if any, to be executed or cancelled.
\item Each Command, is associated with exactly two(2) Titles, one of them belongs to the buyer and the other, to the seller. 
\end{itemize}
Notice that we could use three extra classes extending \textbf{Charge}, namely Purchase, Supply and Tax but other than Supply, a class with a different updateCharge aproach, all of them would not add any attribute or method up to their parent, so we decide not to define classes Purchase and Tax. Lastly, to avoid a huge diagram we omitted the following fields of Customer:\\  
fathername, mothername, address, birthDate, id, tax{\_}id and financialService.
   
\newpage
\subsection{Command Lifecycle}
By reading the description of the system, one might notice that the lifetime of the \emph{command} entity happens to be a realy interesting one. Well, in that case, a common technique is using a state-machine diagram to better visualize all possible outcomes of feature.\\
For a command to be executed, all properties it has, need to first be defined using a form.\\
After a successful submission, meaning all fields contain values of the right type, the command will be reviewed. In case of an approval, it will be executed as soon as possible. In the meantime, it can be cancelled. But, this in only possible in case of an independent command, meaning it has no recursive effect to any previously executed commands, because they can not be reverted now in any way. The diagram follows:\\
\includegraphics[scale=0.325]{state}  

\hypertarget{CmdTrns}{\subsection{Command Transmission}}

\subsubsection{Scenarios}The transmission of a command, although it looks pretty straightforward, is actually a complex one. Let us take a closer look at it step-by-step in order to detect all points where something could go wrong and think about what happens in each of those cases. Steps should look as follows:\\
\begin{enumerate}
\item Customer fills out all attributes of the form
\begin{enumerate}
\item an incomplete form or
\item a field set with a non-matching type or out of range value\\(e.g ``name'' set to an integer, ``time-target'' refers to the past etc.) 
\end{enumerate} 
will take us back to step 1 
\item Customer previews the form
\begin{enumerate}
\item customer decides to modify some field, back to step 1
\end{enumerate}
\item Customer confirms and submits
\begin{enumerate}
\item a network partition occurs, action depends on system availability policy(e.g PACELC, PAVEL etc.) 
\end{enumerate}
\item A.S.C receives request
\begin{enumerate}
\item command execution is not possible, then command is rejected
\end{enumerate}
\item A.S.C approves request, pushes command in the ``To be executed'' priority queue of e-Broker 
\end{enumerate} 

\subsubsection{Activity Diagram}
Once a Customer wishes to execute a command, a form to fill out opens up. It is not until it is complete and contains data of the right type, that the customer can preview it, perhaps request changes and then move on to submit it. At this point the A.S.T receives the submitted form, confirms of receiving it(now the customer is able to parallelly issue another command) and proceed with reviewing it. That will either lead to an approval or a rejection. The exact diagram follows:\\
\includegraphics[scale=0.6]{activity}    

\subsubsection{Detailed Class Diagram}
Now, let us take a closer, more detailed look at all classes interacting during the Command phase as well as the
messages exchanged amongst them.\\
\hspace*{25mm}\includegraphics[scale=0.3]{detailed_classes}\\
As we mentioned above, C.S.D and A.S.E are not classes, but just an instance of one, so they are omitted.\\
The above diagram tells us that for each Command, there is a unique \textbf{Report}, associated with at least one(1) \textbf{Transaction} that involves at least one(1) \textbf{Stock} and is associated with exactly two(2) Titles, one for the buyer and one for the seller.
So, a complete and very detailed class diagramm would look like this\\
\includegraphics[scale=0.25]{classes}

\hypertarget{Asynchronous}{\subsubsection{Sequence Diagram}}
This diagram, in contrast to the Activity one above, analyzes a single scenario from a slightly different scope, using a closer to the implementation approach, intended for the development team to use.\\ 
\includegraphics[scale=0.25]{sequence}\\
As you might did notice, a solid line denotes a request, while a dashed line denotes a response. Lastly, open arrow heads were used for asynchronous requests, notice that a Customer is able to carry on after submission and the System will, obviously, not freeze until the Athens Stock Exchange responds after a Customer's command.

\subsubsection{Communication Diagram}
Now, let us focus on object relationships, rather than the sequence of the messages exchanged amongst them.\\
\includegraphics[scale=0.7]{communication}\\

\section{Structured Design}
In this section, we attempted to better determine the design specifications of e-Broker by creating a Program Structure diagram and using pseudocode to better illustrate the functionality of some vital procedures for this application. 
\subsection{From Dataflow diagram to{\\} Program Structure diagram}
\hspace*{20mm}\includegraphics[scale=0.65]{program-structure}

\subsection{Pseudocode Representation of Units}

\subsubsection{Central Transformation}
\begin{algorithmic}[H]
\STATE \textbf{Procedure M2\_M4}
	\bindent
	\STATE \LocalVar T, Z, K, $\Lambda, P, \Sigma, \Psi, \Pi$ 
	\STATE \Initialize T, Z, K, $\Lambda, P, \Sigma, \Psi, \Pi$ 
	\STATE \Call ExecM1(Z)
	\STATE \Call ExecM2(K, $\Lambda$, P)
	\STATE \Call GetT(T)
	\STATE \Call Calc$\Psi$(Z, K, T, $\Lambda, \Psi$)
	\STATE \Call ExecM3($\Sigma$)
	\STATE \Call Calc$\Pi$(P, $\Sigma, \Pi$)
	\STATE \Call ExecM6($\Psi, \Pi$)
	\eindent
\STATE \textbf{End\_Procedure}
\end{algorithmic}

\subsubsection{M3 Transformation Control Unit}
\begin{algorithmic}[H]
\STATE \textbf{Procedure M3}($\Sigma$: {\color{blue}IN/OUT})
	\bindent
	\STATE \LocalVar $\Delta$, I
	\STATE \Initialize $\Delta$, I 
	\STATE \Call Get$\Delta(\Delta)$
	\STATE \Call CalcI$\Sigma$($\Delta$, I, $\Sigma$)
	\STATE \Call PutI(I)
	\eindent
\STATE \textbf{End\_Procedure}
\end{algorithmic}

\subsubsection{M4 Transformation Calculation Unit}
\begin{algorithmic}[H]
\STATE \textbf{Procedure Calc$\Psi$}(K, $\Lambda$, Z, T {\color{blue} IN}, $\Psi$ {\color{blue}IN/OUT})
	\bindent
	\STATE ...
	\STATE //calculate $\Psi$
	\STATE ...
	\eindent
\STATE \textbf{End\_Procedure}
\end{algorithmic}

\subsubsection{M2 Transformation Presentation Unit}
The presentation unit of a trasnforamtion is the one component, responsible of communicating with all users
and external devices/systems. As one can see, M2 is only communicating with one external device, Source 2.
So, the pseudocode for the Presentation Unit of M2 looks as follows:
\begin{algorithmic}[H]
\STATE \textbf{Procedure GetAB}(A, B: {\color{blue} IN/OUT})
	\bindent
	\STATE ...
	\STATE read A, B from Source\_2
	\STATE ...
	\eindent
\STATE \textbf{End\_Procedure}
\end{algorithmic}

\newpage
\section{Epilogue}

\subsection{Using Structured Analysis}  
\subsubsection{Conclusion}
Looking back at the Structured Analysis chapter, one might realize that
a graphical represantation of a complex system can be very enlightening and informative both for
a user and for a designer. This is what makes Structured Analysis a great tool, not just for software developmnet, but any
kind of system analysis. To be more precise, in our case, e-Broker is pretty complex at first glance,
but data-flow diagrams manage
to give a clear picture of the system flow by dividing each process as much as needed in order to make the whole concept
easily understandable by not only a designer but a common user as well. Therefore, Structured Analysis is definately a tool
that we are going to use to decompose any complex project that we are assigned in the future, isolating each component of
the system. Therefore, deployment becomes an easier and much more enjoyable process.

\subsubsection{Difficulties}
It is rather obvious that the most difficult part of Structured Analysis is designing
a Dataflow Diagram that satisfies the requested level of decomposition, being as precise as required, avoiding
too much detail, but always ensure that all critical points are included.  

\subsubsection{Tips}
\begin{itemize}
 \item \textbf{Preparation} \\
 To better prepare before using Structured Analysis our team suggests to group the parts 
 of this tool as follows:\\
 1. Structured English \\
 2. Dataflow Diagram and Data Dictionary\\ 
 3. Decision Trees and Table\\
 and study the corresponding pairs as a whole due to the fact that they are closely related. 
 
 \item \textbf{Key Points}\\
 We think that it is really important to ensure that the Decision Table and Tree designed are equivalent.
 This is easy to do because of their similarity, so would definately suggest double-checking those
 diagrams before moving on to avoid any subsequent mis-understandings.
 
 \item \textbf{Traps}\\
 Most people do not realize that Structured Analysis is process-driven, so trying
 to describe an object-oriented behavior through it, results in misleading diagrams that might destroy the project.
 Before any kind of design, one must completely understand what the process-driven concept stands for,
 to avoid this deadly trap.    
 
\end{itemize} 

\subsection{Working with UML}

\subsubsection{Conclusion}
Being CS students, using object-oriented languages like, for instance, C++ and Java, is pretty much a common case.
That familiarity helped us absorb the whole usage and syntax of UML rather quickly. This why UML follows the
object oriented concept in the first place. Therefore, we think that UML is a perfect analysis and design tool for anyone coming off an object-oriented backgound.
\subsubsection{Difficulties}
At first glance, UML seemed easy. However, it did not take long for our team to notice that UML is a lot more
complex than it looks. Supporting a variety of different diagrams, it was really hard to
express ourselves and describe the same system over and over again from different aspects. 

\subsubsection{Tips}
After our first experience with UML, we really need to stress the following to anyone interested in
using UML from scratch.  
\begin{itemize}
\item \textbf{Preparation} \\
	First off, we definetely suggest to take a look at 
	\href{https://www.youtube.com/watch?v=UI6lqHOVHic&list=PLUoebdZqEHTxNC7hWPPwLsBmWI0KEhZOd}{\emph{these UML tutorials}} on 			YouTube. To better understand those, try and make the tutorials interactive using a UML tool locally like 
    \href{http://staruml.io/download}{\emph {Star UML}} or online at 
    \href{https://www.lucidchart.com/pages/home}{\emph {Lucidchart}}.  

\item \textbf{Key Points}\\
One thing we realized using UML is that avoiding too much detail is a
pretty good idea. These are high-level tools, allowing us to grasp the concept and philosophy of a system, not
the exact implementation. So, \textbf{KISS}(keep it simple stupid) would be our major advice to anyone new to UML,
trying to be as precise as possible, omitting exhaustive details which are entailed by the rest of the context
and anyone can figure out. 

\item \textbf{Traps}\\
There is a common mis-understanding
when it comes to sequence and communication diagrams.
A sequence diagram is used when the series of events is crucial, while a communication
diagram is used when the messages exchanged between objects is are primary concern.
We think that is very important for one to know the difference before moving on to make on of those.

\end{itemize}


 

\subsection{Structured Analysis vs UML}
To sum things up, both tools can be used in case of a simple user/client to understand how a complex system works taking a really high-level look
at the systems' components. However, UML feels more natural to a software developer because of the recent rise of OOP.
Of course someone working with declarative, especially functional, languages, like Haskell, Structured Analysis
would be a better idea.
Therefore, the appropriate tool is case-dependent and
there are lots of factors to take into consideration before moving on to use one of those, like the size, scope and
other implementation details(e.g programming languages and frameworks) of the to-be-deployed system.
Last but not least, the choice really also depends on who is making the diagrams. UML requires creativity, imagination and
a dose of intuition, while Structured Analysis is pretty straight-forward, in our opinion. 
At the end of the day, we thing that there is no choice to make here.

\end{document}